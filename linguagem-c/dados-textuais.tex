% Options for packages loaded elsewhere
\PassOptionsToPackage{unicode}{hyperref}
\PassOptionsToPackage{hyphens}{url}
\PassOptionsToPackage{dvipsnames,svgnames,x11names}{xcolor}
%
\documentclass[
  letterpaper,
  DIV=11,
  numbers=noendperiod]{scrartcl}

\usepackage{amsmath,amssymb}
\usepackage{iftex}
\ifPDFTeX
  \usepackage[T1]{fontenc}
  \usepackage[utf8]{inputenc}
  \usepackage{textcomp} % provide euro and other symbols
\else % if luatex or xetex
  \usepackage{unicode-math}
  \defaultfontfeatures{Scale=MatchLowercase}
  \defaultfontfeatures[\rmfamily]{Ligatures=TeX,Scale=1}
\fi
\usepackage{lmodern}
\ifPDFTeX\else  
    % xetex/luatex font selection
\fi
% Use upquote if available, for straight quotes in verbatim environments
\IfFileExists{upquote.sty}{\usepackage{upquote}}{}
\IfFileExists{microtype.sty}{% use microtype if available
  \usepackage[]{microtype}
  \UseMicrotypeSet[protrusion]{basicmath} % disable protrusion for tt fonts
}{}
\makeatletter
\@ifundefined{KOMAClassName}{% if non-KOMA class
  \IfFileExists{parskip.sty}{%
    \usepackage{parskip}
  }{% else
    \setlength{\parindent}{0pt}
    \setlength{\parskip}{6pt plus 2pt minus 1pt}}
}{% if KOMA class
  \KOMAoptions{parskip=half}}
\makeatother
\usepackage{xcolor}
\setlength{\emergencystretch}{3em} % prevent overfull lines
\setcounter{secnumdepth}{-\maxdimen} % remove section numbering
% Make \paragraph and \subparagraph free-standing
\makeatletter
\ifx\paragraph\undefined\else
  \let\oldparagraph\paragraph
  \renewcommand{\paragraph}{
    \@ifstar
      \xxxParagraphStar
      \xxxParagraphNoStar
  }
  \newcommand{\xxxParagraphStar}[1]{\oldparagraph*{#1}\mbox{}}
  \newcommand{\xxxParagraphNoStar}[1]{\oldparagraph{#1}\mbox{}}
\fi
\ifx\subparagraph\undefined\else
  \let\oldsubparagraph\subparagraph
  \renewcommand{\subparagraph}{
    \@ifstar
      \xxxSubParagraphStar
      \xxxSubParagraphNoStar
  }
  \newcommand{\xxxSubParagraphStar}[1]{\oldsubparagraph*{#1}\mbox{}}
  \newcommand{\xxxSubParagraphNoStar}[1]{\oldsubparagraph{#1}\mbox{}}
\fi
\makeatother

\usepackage{color}
\usepackage{fancyvrb}
\newcommand{\VerbBar}{|}
\newcommand{\VERB}{\Verb[commandchars=\\\{\}]}
\DefineVerbatimEnvironment{Highlighting}{Verbatim}{commandchars=\\\{\}}
% Add ',fontsize=\small' for more characters per line
\usepackage{framed}
\definecolor{shadecolor}{RGB}{241,243,245}
\newenvironment{Shaded}{\begin{snugshade}}{\end{snugshade}}
\newcommand{\AlertTok}[1]{\textcolor[rgb]{0.68,0.00,0.00}{#1}}
\newcommand{\AnnotationTok}[1]{\textcolor[rgb]{0.37,0.37,0.37}{#1}}
\newcommand{\AttributeTok}[1]{\textcolor[rgb]{0.40,0.45,0.13}{#1}}
\newcommand{\BaseNTok}[1]{\textcolor[rgb]{0.68,0.00,0.00}{#1}}
\newcommand{\BuiltInTok}[1]{\textcolor[rgb]{0.00,0.23,0.31}{#1}}
\newcommand{\CharTok}[1]{\textcolor[rgb]{0.13,0.47,0.30}{#1}}
\newcommand{\CommentTok}[1]{\textcolor[rgb]{0.37,0.37,0.37}{#1}}
\newcommand{\CommentVarTok}[1]{\textcolor[rgb]{0.37,0.37,0.37}{\textit{#1}}}
\newcommand{\ConstantTok}[1]{\textcolor[rgb]{0.56,0.35,0.01}{#1}}
\newcommand{\ControlFlowTok}[1]{\textcolor[rgb]{0.00,0.23,0.31}{\textbf{#1}}}
\newcommand{\DataTypeTok}[1]{\textcolor[rgb]{0.68,0.00,0.00}{#1}}
\newcommand{\DecValTok}[1]{\textcolor[rgb]{0.68,0.00,0.00}{#1}}
\newcommand{\DocumentationTok}[1]{\textcolor[rgb]{0.37,0.37,0.37}{\textit{#1}}}
\newcommand{\ErrorTok}[1]{\textcolor[rgb]{0.68,0.00,0.00}{#1}}
\newcommand{\ExtensionTok}[1]{\textcolor[rgb]{0.00,0.23,0.31}{#1}}
\newcommand{\FloatTok}[1]{\textcolor[rgb]{0.68,0.00,0.00}{#1}}
\newcommand{\FunctionTok}[1]{\textcolor[rgb]{0.28,0.35,0.67}{#1}}
\newcommand{\ImportTok}[1]{\textcolor[rgb]{0.00,0.46,0.62}{#1}}
\newcommand{\InformationTok}[1]{\textcolor[rgb]{0.37,0.37,0.37}{#1}}
\newcommand{\KeywordTok}[1]{\textcolor[rgb]{0.00,0.23,0.31}{\textbf{#1}}}
\newcommand{\NormalTok}[1]{\textcolor[rgb]{0.00,0.23,0.31}{#1}}
\newcommand{\OperatorTok}[1]{\textcolor[rgb]{0.37,0.37,0.37}{#1}}
\newcommand{\OtherTok}[1]{\textcolor[rgb]{0.00,0.23,0.31}{#1}}
\newcommand{\PreprocessorTok}[1]{\textcolor[rgb]{0.68,0.00,0.00}{#1}}
\newcommand{\RegionMarkerTok}[1]{\textcolor[rgb]{0.00,0.23,0.31}{#1}}
\newcommand{\SpecialCharTok}[1]{\textcolor[rgb]{0.37,0.37,0.37}{#1}}
\newcommand{\SpecialStringTok}[1]{\textcolor[rgb]{0.13,0.47,0.30}{#1}}
\newcommand{\StringTok}[1]{\textcolor[rgb]{0.13,0.47,0.30}{#1}}
\newcommand{\VariableTok}[1]{\textcolor[rgb]{0.07,0.07,0.07}{#1}}
\newcommand{\VerbatimStringTok}[1]{\textcolor[rgb]{0.13,0.47,0.30}{#1}}
\newcommand{\WarningTok}[1]{\textcolor[rgb]{0.37,0.37,0.37}{\textit{#1}}}

\providecommand{\tightlist}{%
  \setlength{\itemsep}{0pt}\setlength{\parskip}{0pt}}\usepackage{longtable,booktabs,array}
\usepackage{calc} % for calculating minipage widths
% Correct order of tables after \paragraph or \subparagraph
\usepackage{etoolbox}
\makeatletter
\patchcmd\longtable{\par}{\if@noskipsec\mbox{}\fi\par}{}{}
\makeatother
% Allow footnotes in longtable head/foot
\IfFileExists{footnotehyper.sty}{\usepackage{footnotehyper}}{\usepackage{footnote}}
\makesavenoteenv{longtable}
\usepackage{graphicx}
\makeatletter
\newsavebox\pandoc@box
\newcommand*\pandocbounded[1]{% scales image to fit in text height/width
  \sbox\pandoc@box{#1}%
  \Gscale@div\@tempa{\textheight}{\dimexpr\ht\pandoc@box+\dp\pandoc@box\relax}%
  \Gscale@div\@tempb{\linewidth}{\wd\pandoc@box}%
  \ifdim\@tempb\p@<\@tempa\p@\let\@tempa\@tempb\fi% select the smaller of both
  \ifdim\@tempa\p@<\p@\scalebox{\@tempa}{\usebox\pandoc@box}%
  \else\usebox{\pandoc@box}%
  \fi%
}
% Set default figure placement to htbp
\def\fps@figure{htbp}
\makeatother

\KOMAoption{captions}{tableheading}
\makeatletter
\@ifpackageloaded{tcolorbox}{}{\usepackage[skins,breakable]{tcolorbox}}
\@ifpackageloaded{fontawesome5}{}{\usepackage{fontawesome5}}
\definecolor{quarto-callout-color}{HTML}{909090}
\definecolor{quarto-callout-note-color}{HTML}{0758E5}
\definecolor{quarto-callout-important-color}{HTML}{CC1914}
\definecolor{quarto-callout-warning-color}{HTML}{EB9113}
\definecolor{quarto-callout-tip-color}{HTML}{00A047}
\definecolor{quarto-callout-caution-color}{HTML}{FC5300}
\definecolor{quarto-callout-color-frame}{HTML}{acacac}
\definecolor{quarto-callout-note-color-frame}{HTML}{4582ec}
\definecolor{quarto-callout-important-color-frame}{HTML}{d9534f}
\definecolor{quarto-callout-warning-color-frame}{HTML}{f0ad4e}
\definecolor{quarto-callout-tip-color-frame}{HTML}{02b875}
\definecolor{quarto-callout-caution-color-frame}{HTML}{fd7e14}
\makeatother
\makeatletter
\@ifpackageloaded{caption}{}{\usepackage{caption}}
\AtBeginDocument{%
\ifdefined\contentsname
  \renewcommand*\contentsname{Índice}
\else
  \newcommand\contentsname{Índice}
\fi
\ifdefined\listfigurename
  \renewcommand*\listfigurename{Lista de Figuras}
\else
  \newcommand\listfigurename{Lista de Figuras}
\fi
\ifdefined\listtablename
  \renewcommand*\listtablename{Lista de Tabelas}
\else
  \newcommand\listtablename{Lista de Tabelas}
\fi
\ifdefined\figurename
  \renewcommand*\figurename{Figura}
\else
  \newcommand\figurename{Figura}
\fi
\ifdefined\tablename
  \renewcommand*\tablename{Tabela}
\else
  \newcommand\tablename{Tabela}
\fi
}
\@ifpackageloaded{float}{}{\usepackage{float}}
\floatstyle{ruled}
\@ifundefined{c@chapter}{\newfloat{codelisting}{h}{lop}}{\newfloat{codelisting}{h}{lop}[chapter]}
\floatname{codelisting}{Código}
\newcommand*\listoflistings{\listof{codelisting}{Lista de Listagens}}
\makeatother
\makeatletter
\makeatother
\makeatletter
\@ifpackageloaded{caption}{}{\usepackage{caption}}
\@ifpackageloaded{subcaption}{}{\usepackage{subcaption}}
\makeatother
\makeatletter
\@ifpackageloaded{float}{}{\usepackage{float}}
\makeatother
\makeatletter
\@ifpackageloaded{algorithm}{}{\usepackage{algorithm}}
\makeatother
\makeatletter
\@ifpackageloaded{algxpar}{}{\usepackage[brazilian]{algxpar}}
\makeatother
\makeatletter
\@ifpackageloaded{newunicodechar}{}{\usepackage{newunicodechar}}
\makeatother
\makeatletter
\@ifpackageloaded{tikz}{}{\usepackage{tikz}}
\makeatother
\usetikzlibrary {shapes.geometric}
\newcommand*\invalidchar[1]{\tikz[baseline=(char.base)]{
            \node[fill=black,shape=diamond,draw,inner sep=0pt] (char) {\scriptsize\color{white} #1};}
}
\newunicodechar{�}{\invalidchar{?}}
\newunicodechar{‘}{\'{}}
\newunicodechar{’}{\'{}}

\catcode0=9  % for terminal
\catcode7=9  % for terminal

\ifLuaTeX
\usepackage[bidi=basic]{babel}
\else
\usepackage[bidi=default]{babel}
\fi
\babelprovide[main,import]{brazilian}
% get rid of language-specific shorthands (see #6817):
\let\LanguageShortHands\languageshorthands
\def\languageshorthands#1{}
\usepackage{bookmark}

\IfFileExists{xurl.sty}{\usepackage{xurl}}{} % add URL line breaks if available
\urlstyle{same} % disable monospaced font for URLs
\hypersetup{
  pdftitle={Dados textuais em C},
  pdflang={pt-BR},
  colorlinks=true,
  linkcolor={blue},
  filecolor={Maroon},
  citecolor={Blue},
  urlcolor={Blue},
  pdfcreator={LaTeX via pandoc}}


\title{Dados textuais em C}
\author{}
\date{}

\begin{document}
\maketitle

\floatstyle{plaintop}\restylefloat{algorithm}
\floatname{algorithm}{Algoritmo}

\counterwithin{algorithm}{chapter}

\renewcommand*\contentsname{Índice}
{
\hypersetup{linkcolor=}
\setcounter{tocdepth}{3}
\tableofcontents
}

A linguagem C não é uma linguagem simples para trabalhar com cadeias de
caracteres, as chamadas \emph{strings}. O suporte para textos na
linguagem é feito por constantes inseridas diretamente no código ou
manipuladas em variáveis simples ou compostas do tipo \texttt{char}.

Este capítulo apresenta uma visão inicial das constantes textuais da
linguagem e como fazer referências a essas constantes.

\subsection{Revisitando as constantes
textuais}\label{revisitando-as-constantes-textuais}

No \textbf{?@sec-tipos-de-dados} foram apresentados os principais tipos
de dados da linguagem~C, incluindo os dados textuais, mas ainda é
preciso complementar as informações sobre os tipos literais.

A linguagem C apresenta duas notações para especificar valores textuais.
A primeira é para caracteres únicos, como uma letra ou um símbolo de
pontuação, e é expressa usando aspas simples, como
\texttt{\textquotesingle{}A\textquotesingle{}} ou
\texttt{\textquotesingle{}@\textquotesingle{}}, por exemplo.

O programa seguinte ilustra três escritas usando o formato \texttt{\%c}
(caractere) para apresentar cada valor.

\begin{Shaded}
\begin{Highlighting}[]
\CommentTok{/*}
\CommentTok{Apresentação de caracteres simples}
\CommentTok{Assegura: escrita de caracteres únicos}
\CommentTok{*/}
\PreprocessorTok{\#include }\ImportTok{\textless{}stdio.h\textgreater{}}

\DataTypeTok{int}\NormalTok{ main}\OperatorTok{(}\DataTypeTok{void}\OperatorTok{)} \OperatorTok{\{}
\NormalTok{    printf}\OperatorTok{(}\StringTok{"Uma letra simples: }\SpecialCharTok{\%c}\StringTok{.}\SpecialCharTok{\textbackslash{}n}\StringTok{"}\OperatorTok{,} \CharTok{\textquotesingle{}f\textquotesingle{}}\OperatorTok{);}
\NormalTok{    printf}\OperatorTok{(}\StringTok{"Um símbolo de pontuação: }\SpecialCharTok{\%c}\StringTok{.}\SpecialCharTok{\textbackslash{}n}\StringTok{"}\OperatorTok{,} \CharTok{\textquotesingle{};\textquotesingle{}}\OperatorTok{);}
\NormalTok{    printf}\OperatorTok{(}\StringTok{"O espaço: }\SpecialCharTok{\%c}\StringTok{.}\SpecialCharTok{\textbackslash{}n}\StringTok{"}\OperatorTok{,} \CharTok{\textquotesingle{} \textquotesingle{}}\OperatorTok{);}

    \ControlFlowTok{return} \DecValTok{0}\OperatorTok{;}
\OperatorTok{\}}
\end{Highlighting}
\end{Shaded}

\begin{Shaded}
\begin{Highlighting}[]
\NormalTok{Uma letra simples: f.}
\NormalTok{Um símbolo de pontuação: ;.}
\NormalTok{O espaço:  .}
\end{Highlighting}
\end{Shaded}

A segunda forma de expressar constantes literais usa aspas duplas e
indica uma sequência de caracteres, como \texttt{"linguagem"} e
\texttt{"compilador\ gcc"}. O \texttt{printf} usa a especificação de
formato \texttt{\%s} para substituir cadeias de caracteres.

\begin{Shaded}
\begin{Highlighting}[]
\CommentTok{/*}
\CommentTok{Apresentação de cadeias de caracteres}
\CommentTok{Assegura: escrita de sequências de caracteres}
\CommentTok{*/}
\PreprocessorTok{\#include }\ImportTok{\textless{}stdio.h\textgreater{}}

\DataTypeTok{int}\NormalTok{ main}\OperatorTok{(}\DataTypeTok{void}\OperatorTok{)} \OperatorTok{\{}
\NormalTok{    printf}\OperatorTok{(}\StringTok{"Um texto: }\SpecialCharTok{\%s}\StringTok{.}\SpecialCharTok{\textbackslash{}n}\StringTok{"}\OperatorTok{,} \StringTok{"Era uma vez..."}\OperatorTok{);}
\NormalTok{    printf}\OperatorTok{(}\StringTok{"Texto com uma tabulação: }\SpecialCharTok{\%s\textbackslash{}n}\StringTok{"}\OperatorTok{,} \StringTok{"n =}\SpecialCharTok{\textbackslash{}t}\StringTok{20"}\OperatorTok{);}

    \ControlFlowTok{return} \DecValTok{0}\OperatorTok{;}
\OperatorTok{\}}
\end{Highlighting}
\end{Shaded}

\begin{Shaded}
\begin{Highlighting}[]
\NormalTok{Um texto: Era uma vez....}
\NormalTok{Texto com uma tabulação: n =	20}
\end{Highlighting}
\end{Shaded}

A principal diferença entre um caractere simples e uma cadeia de
caracteres é, além do evidente número de caracteres em si, é a
representação. Enquanto caracteres simples apenas são representados
apenas por um valor, a cadeia usa uma sequência de caracteres simples
seguida pelo terminador \texttt{\textbackslash{}0}, chamado caractere
nulo e que é representado com todos os bits iguais a zero. Na prática,
quando o compilador encontra uma constante como \texttt{"programa"} em
um código, ele armazena esse valor como
\texttt{programa\textbackslash{}0}, ou seja, sempre com um byte nulo
depois do texto em si. Para explicitar um pouco mais,
\texttt{\textquotesingle{}Y\textquotesingle{}} é só o \texttt{Y},
enquanto \texttt{"Y"} é \texttt{Y\textbackslash{}0}.

\subsection{\texorpdfstring{Armazenamento de caracteres simples: o tipo
\texttt{char}}{Armazenamento de caracteres simples: o tipo char}}\label{armazenamento-de-caracteres-simples-o-tipo-char}

Para criar uma variável para armazenar um único caractere deve ser usado
o tipo \texttt{char}. Segue um programa simples ilustrando o conceito.

\begin{Shaded}
\begin{Highlighting}[]
\CommentTok{/*}
\CommentTok{Armazenamento em um char}
\CommentTok{Assegura: escrita dos valores armazenados}
\CommentTok{*/}
\PreprocessorTok{\#include }\ImportTok{\textless{}stdio.h\textgreater{}}

\DataTypeTok{int}\NormalTok{ main}\OperatorTok{(}\DataTypeTok{void}\OperatorTok{)} \OperatorTok{\{}
    \DataTypeTok{char}\NormalTok{ pontuacao }\OperatorTok{=} \CharTok{\textquotesingle{}?\textquotesingle{}}\OperatorTok{;}
    \DataTypeTok{char}\NormalTok{ letra }\OperatorTok{=} \CharTok{\textquotesingle{}A\textquotesingle{}}\OperatorTok{;}

\NormalTok{    printf}\OperatorTok{(}\StringTok{"Qual é a letra}\SpecialCharTok{\%c}\StringTok{ Resposta: }\SpecialCharTok{\%c}\StringTok{!}\SpecialCharTok{\textbackslash{}n}\StringTok{"}\OperatorTok{,}\NormalTok{ pontuacao}\OperatorTok{,}\NormalTok{ letra}\OperatorTok{);}

\NormalTok{    printf}\OperatorTok{(}\StringTok{"Quantidade de bytes em um char: }\SpecialCharTok{\%zu}\StringTok{.}\SpecialCharTok{\textbackslash{}n}\StringTok{"}\OperatorTok{,} \KeywordTok{sizeof}\NormalTok{ letra}\OperatorTok{);}

    \ControlFlowTok{return} \DecValTok{0}\OperatorTok{;}
\OperatorTok{\}}
\end{Highlighting}
\end{Shaded}

\begin{Shaded}
\begin{Highlighting}[]
\NormalTok{Qual é a letra? Resposta: A!}
\NormalTok{Quantidade de bytes em um char: 1.}
\end{Highlighting}
\end{Shaded}

A linguagem C usa o tipo \texttt{char} para caracteres mas, na prática,
trata esse valor como um valor inteiro. Apenas a apresentação com o
\texttt{printf} é que escolhe usar o caractere como valor apresentado.

Seguem dois programas para mostrar que \texttt{char} e \texttt{int} são
muito próximos (mas não iguais!).

\begin{Shaded}
\begin{Highlighting}[]
\CommentTok{/*}
\CommentTok{Apresentação do char como caractere e como inteiro}
\CommentTok{Assegura: apresentação de um char e seu valor associado}
\CommentTok{*/}
\PreprocessorTok{\#include }\ImportTok{\textless{}stdio.h\textgreater{}}

\DataTypeTok{int}\NormalTok{ main}\OperatorTok{(}\DataTypeTok{void}\OperatorTok{)} \OperatorTok{\{}
    \DataTypeTok{char}\NormalTok{ letra }\OperatorTok{=} \CharTok{\textquotesingle{}Z\textquotesingle{}}\OperatorTok{;}
\NormalTok{    printf}\OperatorTok{(}\StringTok{"O caractere }\SpecialCharTok{\%c}\StringTok{ é armazenado usando o valor decimal }\SpecialCharTok{\%d}\StringTok{.}\SpecialCharTok{\textbackslash{}n}\StringTok{"}\OperatorTok{,}
\NormalTok{           letra}\OperatorTok{,}\NormalTok{ letra}\OperatorTok{);}

    \DataTypeTok{int}\NormalTok{ valor }\OperatorTok{=} \DecValTok{90}\OperatorTok{;}
\NormalTok{    printf}\OperatorTok{(}\StringTok{"O valor decimal }\SpecialCharTok{\%d}\StringTok{ pode ser visto como o caractere }\SpecialCharTok{\%c}\StringTok{.}\SpecialCharTok{\textbackslash{}n}\StringTok{"}\OperatorTok{,}
\NormalTok{            valor}\OperatorTok{,}\NormalTok{ valor}\OperatorTok{);}

    \ControlFlowTok{return} \DecValTok{0}\OperatorTok{;}
\OperatorTok{\}}
\end{Highlighting}
\end{Shaded}

\begin{Shaded}
\begin{Highlighting}[]
\NormalTok{O caractere Z é armazenado usando o valor decimal 90.}
\NormalTok{O valor decimal 90 pode ser visto como o caractere Z.}
\end{Highlighting}
\end{Shaded}

Nesse programa, o valor de \texttt{letra}, que é do tipo \texttt{char},
é apresentado usando-se tanto a interpretação como um caractere, com
\texttt{\%c}, quanto como um valor decimal, com \texttt{\%d}. O mesmo é
feito para a variável inteira \texttt{valor}. Nenhum erro ou aviso é
emitido pelo compilador.

\begin{Shaded}
\begin{Highlighting}[]
\CommentTok{/*}
\CommentTok{Apresentação de manipulações curiosas com char}
\CommentTok{Assegura: apresentação de resultados de manipulação}
\CommentTok{*/}
\PreprocessorTok{\#include }\ImportTok{\textless{}stdio.h\textgreater{}}

\DataTypeTok{int}\NormalTok{ main}\OperatorTok{(}\DataTypeTok{void}\OperatorTok{)} \OperatorTok{\{}
    \DataTypeTok{char}\NormalTok{ letra }\OperatorTok{=} \CharTok{\textquotesingle{}A\textquotesingle{}}\OperatorTok{;}
\NormalTok{    printf}\OperatorTok{(}\StringTok{"A letra atual é }\SpecialCharTok{\%c}\StringTok{. "}\OperatorTok{,}\NormalTok{ letra}\OperatorTok{);}
\NormalTok{    printf}\OperatorTok{(}\StringTok{"Depois dela vem o }\SpecialCharTok{\%c}\StringTok{.}\SpecialCharTok{\textbackslash{}n}\StringTok{"}\OperatorTok{,}\NormalTok{ letra }\OperatorTok{+} \DecValTok{1}\OperatorTok{);}

\NormalTok{    printf}\OperatorTok{(}\StringTok{"}\SpecialCharTok{\textbackslash{}n}\StringTok{Alfabeto minúsculo: "}\OperatorTok{);}
    \ControlFlowTok{for} \OperatorTok{(}\DataTypeTok{char}\NormalTok{ letra }\OperatorTok{=} \CharTok{\textquotesingle{}a\textquotesingle{}}\OperatorTok{;}\NormalTok{ letra }\OperatorTok{\textless{}=} \CharTok{\textquotesingle{}z\textquotesingle{}}\OperatorTok{;}\NormalTok{ letra}\OperatorTok{++)}
\NormalTok{        printf}\OperatorTok{(}\StringTok{"}\SpecialCharTok{\%c}\StringTok{"}\OperatorTok{,}\NormalTok{ letra}\OperatorTok{);}
\NormalTok{    printf}\OperatorTok{(}\StringTok{"}\SpecialCharTok{\textbackslash{}n}\StringTok{"}\OperatorTok{);}

\NormalTok{    printf}\OperatorTok{(}\StringTok{"Alfabeto maiúsculo: "}\OperatorTok{);}
    \ControlFlowTok{for} \OperatorTok{(}\DataTypeTok{char}\NormalTok{ letra }\OperatorTok{=} \CharTok{\textquotesingle{}A\textquotesingle{}}\OperatorTok{;}\NormalTok{ letra }\OperatorTok{\textless{}=} \CharTok{\textquotesingle{}Z\textquotesingle{}}\OperatorTok{;}\NormalTok{ letra}\OperatorTok{++)}
\NormalTok{        printf}\OperatorTok{(}\StringTok{"}\SpecialCharTok{\%c}\StringTok{"}\OperatorTok{,}\NormalTok{ letra}\OperatorTok{);}
\NormalTok{    printf}\OperatorTok{(}\StringTok{"}\SpecialCharTok{\textbackslash{}n}\StringTok{"}\OperatorTok{);}

\NormalTok{    printf}\OperatorTok{(}\StringTok{"}\SpecialCharTok{\textbackslash{}n}\StringTok{A quantidade de letras de }\SpecialCharTok{\%c}\StringTok{ até }\SpecialCharTok{\%c}\StringTok{ é }\SpecialCharTok{\%d}\StringTok{.}\SpecialCharTok{\textbackslash{}n}\StringTok{"}\OperatorTok{,}
           \CharTok{\textquotesingle{}H\textquotesingle{}}\OperatorTok{,} \CharTok{\textquotesingle{}T\textquotesingle{}}\OperatorTok{,} \CharTok{\textquotesingle{}T\textquotesingle{}} \OperatorTok{{-}} \CharTok{\textquotesingle{}H\textquotesingle{}} \OperatorTok{+} \DecValTok{1}\OperatorTok{);}

\NormalTok{    printf}\OperatorTok{(}\StringTok{"}\SpecialCharTok{\textbackslash{}n}\StringTok{A letra que fica no meio de }\SpecialCharTok{\%c}\StringTok{ e }\SpecialCharTok{\%c}\StringTok{ é }\SpecialCharTok{\%c}\StringTok{.}\SpecialCharTok{\textbackslash{}n}\StringTok{"}\OperatorTok{,}
           \CharTok{\textquotesingle{}G\textquotesingle{}}\OperatorTok{,} \CharTok{\textquotesingle{}K\textquotesingle{}}\OperatorTok{,} \OperatorTok{(}\CharTok{\textquotesingle{}G\textquotesingle{}} \OperatorTok{+} \CharTok{\textquotesingle{}K\textquotesingle{}}\OperatorTok{)} \OperatorTok{/} \DecValTok{2}\OperatorTok{);}

    \ControlFlowTok{return} \DecValTok{0}\OperatorTok{;}
\OperatorTok{\}}
\end{Highlighting}
\end{Shaded}

\begin{Shaded}
\begin{Highlighting}[]
\NormalTok{A letra atual é A. Depois dela vem o B.}

\NormalTok{Alfabeto minúsculo: abcdefghijklmnopqrstuvwxyz}
\NormalTok{Alfabeto maiúsculo: ABCDEFGHIJKLMNOPQRSTUVWXYZ}

\NormalTok{A quantidade de letras de H até T é 13.}

\NormalTok{A letra que fica no meio de G e K é I.}
\end{Highlighting}
\end{Shaded}

Reforçando a associação de \texttt{char} com um valor inteiro, o
programa exemplifica operações como \texttt{letra\ +\ 1} e incrementos
como \texttt{letra++}. Adicionalmente há ainda uma expressão
interessante:
\texttt{(\textquotesingle{}G\textquotesingle{}\ +\ \textquotesingle{}K\textquotesingle{})\ /\ 2},
que soma \texttt{\textquotesingle{}G\textquotesingle{}} (71) com de
\texttt{\textquotesingle{}K\textquotesingle{}} (75) e divide o resultado
por 2 (divisão inteira), resultando em 73, que é
\texttt{\textquotesingle{}I\textquotesingle{}}.

\begin{tcolorbox}[enhanced jigsaw, colbacktitle=quarto-callout-warning-color!10!white, left=2mm, coltitle=black, opacitybacktitle=0.6, colframe=quarto-callout-warning-color-frame, toprule=.15mm, bottomrule=.15mm, bottomtitle=1mm, rightrule=.15mm, opacityback=0, arc=.35mm, breakable, leftrule=.75mm, toptitle=1mm, title=\textcolor{quarto-callout-warning-color}{\faExclamationTriangle}\hspace{0.5em}{Curiosidade}, titlerule=0mm, colback=white]

Cada letra possui um valor específico associado a ela, assim como cada
um dos outros caracteres. Por exemplo,
\texttt{\textquotesingle{}A\textquotesingle{}} é o 65,
\texttt{\textquotesingle{}\textbackslash{}n\textquotesingle{}} é o 10 e
\texttt{\textquotesingle{}\ \textquotesingle{}} (espaço) é o 32.

Como esperado, se \texttt{\textquotesingle{}A\textquotesingle{}} é 65,
\texttt{\textquotesingle{}B\textquotesingle{}} é 66,
\texttt{\textquotesingle{}C\textquotesingle{}} é 67 e assim por diante.
Portanto, é possível verificar se \texttt{letra1\ \textless{}=\ letra2},
pois uma comparação de inteiros é feita.

Há um ``porém'' nessa história:
\texttt{\textquotesingle{}Z\textquotesingle{}} é 90, mas
\texttt{\textquotesingle{}a\textquotesingle{}} é 97. Na realidade, as
minúsculas se iniciam no 97 e vão até o 122. Isso gera uma situação
confusa, pois C tem certeza que
\texttt{\textquotesingle{}a\textquotesingle{}\ \textgreater{}\ \textquotesingle{}Z\textquotesingle{}}
é verdadeiro.

\end{tcolorbox}

\begin{tcolorbox}[enhanced jigsaw, colbacktitle=quarto-callout-tip-color!10!white, left=2mm, coltitle=black, opacitybacktitle=0.6, colframe=quarto-callout-tip-color-frame, toprule=.15mm, bottomrule=.15mm, bottomtitle=1mm, rightrule=.15mm, opacityback=0, arc=.35mm, breakable, leftrule=.75mm, toptitle=1mm, title=\textcolor{quarto-callout-tip-color}{\faLightbulb}\hspace{0.5em}{Dica}, titlerule=0mm, colback=white]

Embora haja um prejuízo quanto á clareza, o tipo \texttt{char} pode ser
usado como um ``pequeno inteiro'' de um byte de comprimento. Dessa
forma, usando o primeiro bit para indicar o sinal, uma variável desse
tipo guardaria valores de -128 a 127. Caso necessário, pode-se usar a
declaração de um \texttt{unsigned\ char} para ter valores de 0 até 255.

Para se ter programas mais claros, entretanto, mesmo nesse caso é melhor
usar os tipos de comprimento fixo definidos em \texttt{stdint.h}. Assim,
para se ter um inteiro com sinal de oito bits, o tipo seria
\texttt{int8\_t}, e poderia ser usado o \texttt{uint8\_t} para o mesmo
comprimento sem o sinal. O tipo \texttt{char} deve ser deixado de fato
apenas para caracteres. O uso de \texttt{stdint.h} é abordado no
\textbf{?@sec-arquivos-binarios}.

\end{tcolorbox}

\subsubsection{Caracteres acentuados}\label{sec-caracteres-acentuados}

Na década de 1960 cada sistema computacional usava sua própria tabela
para associar um dado caractere a um valor numérico. Para que os
diferentes sistemas pudessem trocar informações, foi elaborada em 1964
uma codificação padronizada denominada \emph{American Standard Code for
Information Interchange}, ou tabela ASCII. Essa tabela definia tanto os
caracteres de controle (\texttt{\textbackslash{}n} ou
\texttt{\textbackslash{}t}, por exemplo) quanto os caracteres legíveis
(letras, dígitos e pontuação).

Na prática, tanto os caracteres de controle quanto os símbolos legíveis
significativos poderiam ser representados com apenas sete bits. Em
sistemas com palavras de oito bits, o primeiro bit era usualmente zero,
de forma que praticamente metade dos bytes não tinham uso. O ponto em
questão é que o conjunto de caracteres era baseado na lingua inglesa, na
qual acentuações ou outros símbolos de outras línguas não eram
incluídos, como \texttt{á}, \texttt{ã}, \texttt{Ç}, \texttt{ß} (alemão)
ou \texttt{č} (esloveno). Os bytes não usados na codificação ASCII
(aqueles cujos bits começavam com 1) eram usados para esse fim, cada
sistema usando uma codificação específica e particular para atender suas
necessidades.

\index{Unicode} \index{UTF-8} Em grande parte dos sistemas atuais é
empregado o Unicode\footnote{Unicode Consortium:
  \url{https://home.unicode.org}.}, que se propõe a ter representações
para todas as línguas do planeta e usa com frequência a codificação
UTF-8 para representar os símbolos (letras, ideogramas, emojis) em
bytes. Como exemplo, o símbolo monetário do Euro é designado por
\texttt{U+20AC} no Unicode e usa a sequência de bytes \texttt{E282AC}
para representação em UTF-8.

O problema que surge dessa representação é que UTF-8 usa uma quantidade
de bytes variável conforme o símbolo. Os caracteres ASCII usam um único
byte e possuem representação igual, o que mantém a compatibilidade.
Outros símbolos, porém usam dois, três ou até quatro bytes, o que torna
impossível armazená-los em uma variável \texttt{char}.

\begin{Shaded}
\begin{Highlighting}[]
\CommentTok{/*}
\CommentTok{Incompatibilidade de símbolos Unicode/UTF{-}8}
\CommentTok{*/}
\PreprocessorTok{\#include }\ImportTok{\textless{}stdio.h\textgreater{}}

\DataTypeTok{int}\NormalTok{ main}\OperatorTok{(}\DataTypeTok{void}\OperatorTok{)} \OperatorTok{\{}
    \DataTypeTok{char}\NormalTok{ c }\OperatorTok{=} \CharTok{\textquotesingle{}é\textquotesingle{}}\OperatorTok{;}
\NormalTok{    printf}\OperatorTok{(}\StringTok{"c = }\SpecialCharTok{\%c}\StringTok{.}\SpecialCharTok{\textbackslash{}n}\StringTok{"}\OperatorTok{,}\NormalTok{ c}\OperatorTok{);}

    \ControlFlowTok{return} \DecValTok{0}\OperatorTok{;}
\OperatorTok{\}}
\end{Highlighting}
\end{Shaded}

\begin{Shaded}
\begin{Highlighting}[]
\NormalTok{main.c: In function ‘main’:}
\NormalTok{main.c:7:14: warning: multi{-}character character constant [{-}Wmultichar]}
\NormalTok{    7 |     char c = \textquotesingle{}é\textquotesingle{};}
\NormalTok{      |              \^{}\textasciitilde{}\textasciitilde{}}
\NormalTok{main.c:7:14: warning: overflow in conversion from ‘int’ to ‘char’ }
\NormalTok{changes value from ‘50089’ to ‘{-}87’ [{-}Woverflow]}
\end{Highlighting}
\end{Shaded}

\begin{Shaded}
\begin{Highlighting}[]
\NormalTok{c = �.}
\end{Highlighting}
\end{Shaded}

A falha na compilação acima é que o caractere \texttt{é}
(\texttt{U+00E9}) é codificado em dois bytes, \texttt{C3} e \texttt{A9},
dos quais apenas o valor \texttt{A9} (decimal 169) é guardado na
variável \texttt{c}. Esse valor 169, sozinho, não representa um
caractere UTF-8 válido.

Esse código fonte, armazenado no arquivo \texttt{acentuacao.c}, está
codificado com UTF-8, como se indica com o comando \texttt{file}.

\begin{Shaded}
\begin{Highlighting}[]
\NormalTok{$ file acentuacao.c}
\NormalTok{acentuacao.c: C source, Unicode text, UTF{-}8 text}
\end{Highlighting}
\end{Shaded}

Existe a codificação de caracteres ISO-8859, conhecida como
\emph{latin1}, que inclui os caracteres latinos acentuados e que usa
apenas um byte por caractere. O comando \texttt{iconv} pode ser usado
para criar um novo arquivo fonte (\texttt{acentuacao-latin1.c}) com essa
codificação, conforme segue.

\begin{Shaded}
\begin{Highlighting}[]
\NormalTok{$ iconv {-}f utf8 {-}t latin1 acentuacao.c \textgreater{} acentuacao{-}latin1.c}
\NormalTok{$ file acentuacao{-}latin1.c}
\NormalTok{acentuacao{-}latin1.c: C source, ISO{-}8859 text}
\end{Highlighting}
\end{Shaded}

Como o caractere \texttt{é} usado no código fonte agora possui um único
byte, ele pode ser guardado em um \texttt{char} e a compilação ocorre
sem problemas.

\begin{Shaded}
\begin{Highlighting}[]
\NormalTok{$ gcc {-}Wall {-}pedantic {-}std=c17 acentuacao{-}latin1.c}
\end{Highlighting}
\end{Shaded}

Ao executar o programa, como a saída produzida é ISO-8859, ela tem que
ser convertida de volta para UTF-8 para que o terminal a exiba
corretamente.

\begin{Shaded}
\begin{Highlighting}[]
\NormalTok{$ ./a.out | iconv {-}f latin1 {-}t utf8}
\NormalTok{c = é.}
\end{Highlighting}
\end{Shaded}

Nos programas exemplificados neste texto, simplesmente são evitados os
casos em que um \texttt{char} armazenará um caractere Unicode com mais
que um byte, pois todas as codificações de caractere usam UTF-8. Para
efetivamente usar caracteres de múltiplos bytes, C disponibiliza uma
série de funções em \texttt{wchar.c}, as quais lidam com os ``caracteres
largos'' (\emph{wide characters}). Porém, o uso dessas funções não é
tratado neste livro.

\subsection{Acesso às cadeias de caracteres
constantes}\label{sec-ponteiros-para-strings-constantes}

Nesta seção é apresentada uma visão básica sobre cadeias de caracteres
em~C, sendo que a manipulação de variáveis com conteúdo textual é
coberta pelo \textbf{?@sec-manipulacao-de-dados-textuais}.

Em programas escritos em C, cadeias de caracteres são indicadas entre
aspas duplas e, internamente, é acrescido um byte nulo para indicar seu
fim. Quando uma constante literal é parte de um programa, ela é incluída
na seção de dados do arquivo executável.

Um exemplo trivial dessa inclusão pode ser vista com um código simples,
como o seguinte.

\begin{Shaded}
\begin{Highlighting}[]
\CommentTok{/*}
\CommentTok{Apresentação de mensagens simples}
\CommentTok{Assegura: apresentação de duas mensagens na tela}
\CommentTok{*/}
\PreprocessorTok{\#include }\ImportTok{\textless{}stdio.h\textgreater{}}

\DataTypeTok{int}\NormalTok{ main}\OperatorTok{(}\DataTypeTok{void}\OperatorTok{)} \OperatorTok{\{}
\NormalTok{    printf}\OperatorTok{(}\StringTok{"Bom dia!}\SpecialCharTok{\textbackslash{}n}\StringTok{"}\OperatorTok{);}
\NormalTok{    printf}\OperatorTok{(}\StringTok{"Boa noite.}\SpecialCharTok{\textbackslash{}n}\StringTok{"}\OperatorTok{);}

    \ControlFlowTok{return} \DecValTok{0}\OperatorTok{;}
\OperatorTok{\}}
\end{Highlighting}
\end{Shaded}

\begin{Shaded}
\begin{Highlighting}[]
\NormalTok{Bom dia!}
\NormalTok{Boa noite.}
\end{Highlighting}
\end{Shaded}

Esse programa gera um executável denominado \texttt{a.out}, como
observado pelos comandos que seguem.

\begin{Shaded}
\begin{Highlighting}[]
\NormalTok{$ ls {-}l a.out}
\NormalTok{{-}rwxrwxr{-}x 1 jander jander 16136 fev 17 14:34 a.out}
\NormalTok{$ file a.out}
\NormalTok{a.out: ELF 64{-}bit LSB pie executable, x86{-}64, version 1 (SYSV), dynamically }
\NormalTok{linked, interpreter /lib64/ld{-}linux{-}x86{-}64.so.2, }
\NormalTok{BuildID[sha1]=a6b3fbd1826fd3417d0b649b1bca180bbeb2aae7, for GNU/Linux 3.2.0, }
\NormalTok{not stripped}
\end{Highlighting}
\end{Shaded}

O comando \texttt{strings} usado na sequência mostra as cadeias de
caracteres detectadas no arquivo \texttt{a.out} (o \texttt{egrep} é
usado para remover outras linhas, de forma a reduzir o tamanho da
saída).

\begin{Shaded}
\begin{Highlighting}[]
\NormalTok{$ strings a.out | egrep {-}v \textquotesingle{}(\^{}\textbackslash{}.|\^{}\_)\textquotesingle{}}
\NormalTok{/lib64/ld{-}linux{-}x86{-}64.so.2}
\NormalTok{puts}
\NormalTok{printf}
\NormalTok{libc.so.6}
\NormalTok{GLIBC\_2.2.5}
\NormalTok{GLIBC\_2.34}
\NormalTok{PTE1}
\NormalTok{u+UH}
\NormalTok{\%c\%s\%c}
\NormalTok{Bom dia!}
\NormalTok{Boa noite.}
\NormalTok{9*3$"}
\NormalTok{GCC: (Ubuntu 13.3.0{-}6ubuntu2\textasciitilde{}24.04) 13.3.0}
\NormalTok{Scrt1.o}
\NormalTok{crtstuff.c}
\NormalTok{deregister\_tm\_clones}
\NormalTok{completed.0}
\NormalTok{frame\_dummy}
\NormalTok{main.c}
\NormalTok{puts@GLIBC\_2.2.5}
\NormalTok{printf@GLIBC\_2.2.5}
\NormalTok{user\_input\_time}
\NormalTok{print\_user\_input}
\NormalTok{main}
\end{Highlighting}
\end{Shaded}

Como pode ser observado, \texttt{"Bom\ dia!"} e \texttt{"Boa\ noite."}
fazem parte dos literais fisicamente presentes no arquivo.

Com base no fato de que as constantes fazem parte do arquivo executável
e que quando o programa é colocado em execução elas são também
carregadas para a memória, o programa seguinte mostra como fazer
referência a essas cadeias.

\begin{Shaded}
\begin{Highlighting}[]
\CommentTok{/*}
\CommentTok{Exemplificação de referência a constantes literais presentes em um programa}
\CommentTok{Assegura: apresentação de textos}
\CommentTok{*/}
\PreprocessorTok{\#include }\ImportTok{\textless{}stdio.h\textgreater{}}
\PreprocessorTok{\#include }\ImportTok{\textless{}stdlib.h\textgreater{}}

\DataTypeTok{int}\NormalTok{ main}\OperatorTok{(}\DataTypeTok{void}\OperatorTok{)} \OperatorTok{\{}
    \DataTypeTok{char} \OperatorTok{*}\NormalTok{texto1 }\OperatorTok{=} \StringTok{"Um primeiro texto"}\OperatorTok{;}
    \DataTypeTok{char} \OperatorTok{*}\NormalTok{texto2 }\OperatorTok{=} \StringTok{"Texto número 2"}\OperatorTok{;}

    \DataTypeTok{char} \OperatorTok{*}\NormalTok{texto\_selecionado}\OperatorTok{;}
    \ControlFlowTok{if} \OperatorTok{(}\NormalTok{rand}\OperatorTok{()} \OperatorTok{\%} \DecValTok{100} \OperatorTok{\textless{}} \DecValTok{50}\OperatorTok{)}
\NormalTok{        texto\_selecionado }\OperatorTok{=}\NormalTok{ texto1}\OperatorTok{;}
    \ControlFlowTok{else}
\NormalTok{        texto\_selecionado }\OperatorTok{=}\NormalTok{ texto2}\OperatorTok{;}

\NormalTok{    printf}\OperatorTok{(}\StringTok{"1) }\SpecialCharTok{\%s\textbackslash{}n}\StringTok{"}\OperatorTok{,}\NormalTok{ texto1}\OperatorTok{);}
\NormalTok{    printf}\OperatorTok{(}\StringTok{"2) }\SpecialCharTok{\%s\textbackslash{}n}\StringTok{"}\OperatorTok{,}\NormalTok{ texto2}\OperatorTok{);}
\NormalTok{    printf}\OperatorTok{(}\StringTok{"}\SpecialCharTok{\textbackslash{}n}\StringTok{Sorteado: }\SpecialCharTok{\%s\textbackslash{}n}\StringTok{"}\OperatorTok{,}\NormalTok{ texto\_selecionado}\OperatorTok{);}

    \ControlFlowTok{return} \DecValTok{0}\OperatorTok{;}
\OperatorTok{\}}
\end{Highlighting}
\end{Shaded}

\begin{Shaded}
\begin{Highlighting}[]
\NormalTok{1) Um primeiro texto}
\NormalTok{2) Texto número 2}

\NormalTok{Sorteado: Texto número 2}
\end{Highlighting}
\end{Shaded}

Neste programa há dois textos importantes:
\texttt{"Um\ primeiro\ texto"} e \texttt{"Texto\ número\ 2"}. Essas duas
constantes textuais ficam em algum lugar do programa. Existem também as
duas variáveis \texttt{texto1} e \texttt{texto2}, que são declaradas
como sendo do tipo \texttt{char\ *}. O asterisco, nesse contexto, indica
que as variáveis são referências aos textos e não que sejam um
\texttt{char} comum. Dessa forma, \texttt{texto1} está referenciando a
constante \texttt{"Um\ primeiro\ texto"}, por exemplo.

Essas variáveis que guardam referências às coisas recebem, em
programação, o nome de ponteiros\footnote{Esse tema é tratado em mais
  detalhes no \textbf{?@sec-enderecamento-de-memoria-e-ponteiros}.}.
Como elas são usadas apenas como referência a um texto já existente, não
podem ser diretamente usadas para outras manipulações, como para guardar
um valor digitado pelo usuário ou tentar modificar a constante
referenciada.

No jargão computacional, diz-se que ``\texttt{texto1} aponta para a
constante \texttt{"Um\ primeiro\ texto"}'', da mesma forma que
\texttt{texto2} aponta para a constante \texttt{"Texto\ número\ 2"}.

No código há ainda a variável \texttt{texto\_selecionado}, também do
tipo \texttt{char\ *}, cujo valor inicialmente é indefinido (lixo). A
expressão \texttt{rand()\ \%\ 100} resulta em um valor (pseudo)aleatório
de 0 a 99. Assim, a referência ser guardada em
\texttt{texto\_selecionado} depende desse valor aleatório, com
praticamente 50\% de chance para cada caso. Se o valor aleatório for
menor que 50, \texttt{texto\_selecionado} guarda a mesma referência
guardada em \texttt{texto1}, ou seja, ela também aponta para a constante
\texttt{"Um\ primeiro\ texto"}

Nas chamadas a \texttt{printf}, a especificação de formato \texttt{\%s}
apresenta o texto apontado por cada variável. Desse modo,
\texttt{printf("2)\ \%s\textbackslash{}n",\ texto2);} significa mostre o
texto que \texttt{texto2} está apontando.

O uso desse recurso é relativamente simples de der feito, porém muito
limitado, visto que apenas aceita mudar para qual constante cada
variável aponta.

Para finalizar, é interessante ver que esse recurso foi empregado na
implementação do Algoritmo~\ref{alg-classificacao-triangulo-lados}.




\end{document}
